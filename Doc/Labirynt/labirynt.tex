\documentclass{article}
\usepackage{xcolor}
\usepackage{listings}
\usepackage{polski}
\usepackage[utf8]{inputenc}
\title{Labirynt}
\author{Jan Wojciechowski}
\date{\today}
\lstset
{
    basicstyle=\tt\scriptsize,
    identifierstyle=\color{blue},
    commentstyle=\color{red},
    breaklines=true,
    backgroundcolor=\color{yellow!30},
    numbers=none,
    language=[Sharp]C
}
\begin{document}
\maketitle
\section{Wstęp}
Poniższy dokument zawiera informacje o sposobie implementacji labiryntu oraz jego parsowaniu. Strona tytułowa zawiera datę ostatniej modyfikacji tej dokumentacji, w razie zdezaktualizowania proszę powiadomić osobę odpowiedzialną za zmiany i/lub osobę odpowiedzialną za rozwój tej funkcjonalności.
\section{Założenia teoretyczne}
Sam labirynt jest planszą w kształcie prostokąta wypełnioną pokojami i korytarzami. Plansza ta zbudowana jest z kwadratowych sektorów, na których mogą znajdować się różne obiekty, ściany, NPC oraz gracz. Rozmiary planszy są nieograniczone. Każdy labirynt wczytywany jest z przygotowanego wcześniej pliku bitmapy. Kolory mogące wystąpić na bitmapie to:
\begin{description}
  \item[\#000000] \hfill \\
    Korytarz
  \item[\#808080] \hfill \\
    Pomieszczenie
  \item[\#FFFF00] \hfill \\
    Obiekt
  \item[\#E02040] \hfill \\
    NPC
\end{description}
Pozostałe kolory są pomijane.
\section{Stage}
Stage jest implementacją stanu labiryntu. Jest on czymś na kształt ramki, w której umieszczane są pokoje, korytarze etc., lecz nie ściany, one opisane są niżej. Od strony implementacyjnej jest to zbiór tablic dwuwymiarowych bitów, które mówią o tym, gdzie dany obiekt się znajduje na planszy i jaki obszar zajmuje.
\section{StageStructure}
StageStructure jest klasą umożliwiającą wtórne parsowanie labiryntu celem uzyskania struktury ścian i podłóg. Klasa ta posiada 3 właściwości:
\begin{description}
  \item[Count] \hfill \\
    Zwraca sumę ilości bloków ścian i podłóg
  \item[Walls] \hfill \\
    Obiekt reprezentujący kolekcję bloków ścian
  \item[Floor] \hfill \\
    Obiekt reprezentujący kolekcję podłóg
\end{description}
\subsection{Walls}
Obiekt ten składa się z następujących kolekcji:
\begin{description}
  \item[WallsUp] \hfill \\
    Kolekcja ścian patrzących w dół
  \item[WallsDown] \hfill \\
    Kolekcja ścian patrzących w górę
  \item[WallsLeft] \hfill \\
    Kolekcja ścian patrzących w prawo
  \item[WallsRight] \hfill \\
    Kolekcja ścian patrzących w lewo
  \item[CornersUpperLeft] \hfill \\
    Kolekcja kątów ostrych patrzących w prawy dolny róg
  \item[CornersUpperRight] \hfill \\
    Kolekcja kątów ostrych patrzących w lewy dolny róg
  \item[CornersLowerLeft] \hfill \\
    Kolekcja kątów ostrych patrzących w prawy górny róg
  \item[CornersLowerRight] \hfill \\
    Kolekcja kątów ostrych patrzących w lewy górny róg
\end{description}
Warto zauważyć, że każda ze ścian ulokowana jest \textit{wewnątrz} pomieszczenia, tak więc pomieszczenie składające się z jednego bloku będzie posiadało cztery rogi, przy czym wszystkie one będą ulokowane w tym samym bloku, to jest tam, gdzie znajduje się jedyny kwadrat pomieszczenia.
\subsection{Floors}
Obiekt składa się z kolekcji punktów, na których powinna znaleźć się podłoga.
%\lstinputlisting{project.cs}
\end{document}